\documentclass[11pt]{article}
\usepackage{amsmath}
\usepackage{amsfonts}
\usepackage{amssymb}
\usepackage{amsthm}
\usepackage{physics}
\usepackage{float}
\usepackage[margin=1.0in]{geometry}
\usepackage{enumerate}
\usepackage{indentfirst}
\usepackage{graphicx}

\def\vecsign{\mathchar"017E}
\def\dvecsign{\smash{\stackon[-1.95pt]{\vecsign}{\rotatebox{180}{$\vecsign$}}}}
% \def\dvec#1{\def\useanchorwidth{T}\stackon[-4.2pt]{#1}{\,\dvecsign}}
\def\dvec#1{\overset{\leftrightarrow}{#1}}
\usepackage{stackengine}
\stackMath

\newtheorem{theorem}{Theorem}[section]
\newtheorem{lemma}[theorem]{Lemma}
\numberwithin{equation}{section}
\DeclareMathOperator{\sgn}{sgn}

\begin{document}
    \title{Title (TBD)}
    \author{Wojciech Śmiałek}
    \date{}
    \maketitle

    % \begin{abstract}
    % \end{abstract}

    \section{The Klein - Gordon equation}
    Schrodinger equation plays a fundamental role in quantum mechanics. It follows from the postulates 
    of quantum mechanics, that a time evolution of a state vector in an isolated system ought to be described by a unitary transformation. 
    That, by Stone's theorem \cite{stone} implies the existence of a self-adjoint Hamiltonian operator, which determines
    the time evolution of a state vector, according to
    \begin{equation}
     i \hbar \partial_t |\psi\rangle_t = \hat H |\psi\rangle_t \label{sch1}.
    \end{equation}
    In this sense, the Schrodinger equation is fundamental to quantum mechanics, and the existence of a Hamiltonian operator is necessary for a well-defined quantum mechanical system.
    The general Schrodinger equation (\ref{sch1}), however, is not equivalent to a wave equation governing the dynamics of a single nonrelativistic particle, often also simply called the Schrodinger equation:
    \begin{equation}
     i \hbar \partial_t \Psi(\vec x, t) = \left[ -\frac{\hbar^2}{2m} \bigtriangleup + V(\vec x,t) \right] \Psi(\vec x, t) \label{sch2}.
    \end{equation} 
    This equation cannot be universally valid, due to its lack of Lorentz covariance.
    The description of a free relativistic particle necessitates the use of other Lorentz covariant wave equations,
    the forms of which fundamentally depend on the spin of the described particle. 
    Nonetheless, for any of these relativistic descriptions, the existence of a Hamiltonian acting 
    on a Hilbert space of states is necessary for the quantum description to be meaningful.
    
    Virtually all of the relativistic descriptions of noninteracting particles are rooted in the free
    Klein - Gordon equation.
    Klein - Gordon equation is essentially a quantized relativistic
    dispersion relation $E^2 = p^2c^2 + m^2c^4$, supplied with the de Broglie relation for the momentum and the wave vector of a matter wave, $p^\mu = \hbar k^\mu$.
    The four-laplacian operator acting on a matter wave produces a
    Lorentz scalar, the magnitude of which is determined by the aforementioned relations:
    \begin{gather}
     f_k = A e^{-ikx}\\
     \partial^\mu f_{k} = -i k^\mu f_{k} = -\frac{i}{\hbar} p^\mu f_{k} = -\frac{i m}{\hbar} U^\mu f_{k}\\
     \partial_\mu \partial^\mu f_{k} = \frac{m^2}{\hbar^2} U_\mu U^\mu f_{k} = \frac{m^2 c^2}{\hbar^2} f_{k}, \label{relation}
    \end{gather}
    where $U^\mu = \frac{dx^\mu}{d\tau}$ is a four-velocity and $U^\mu U_\mu = c^2$ follows directly from the kinematics of special relativity.
    Conventionally, $x$ will represent a four-vector, while $\vec x$ will represent a three vector.
    This relation, applied to a spin-0, or a single component scalar field, is called the free Klein - Gordon equation:
    \begin{equation}
     \left( \partial_\mu \partial^\mu - \frac{m^2 c^2}{\hbar^2} \right) \phi(x) = 0. \label{kg}
    \end{equation}
    The field $\phi(x)$ is in general a complex scalar. Real Klein-Gordon fields can be studied as well, but in this work, we shall focus solely on the complex case.
    From this point onwards, we shall also use the natural system of units in which $c = \hbar = 1$. % unless explicitly stated otherwise.
    The same relation that serves as a ground for the Klein-Gordon equation, can also be found in the vacuum Maxwell equations for potentials in the Lorentz gauge:
    \begin{gather}
     \partial_\nu \partial^\nu A^{\mu} = 0.
    \end{gather}
    Here, the mass of the field is equal to zero and the field itself is a vector or a spin-1 real field.
    In this way, classical electrodynamics in a vacuum is essentially a relativistic wave equation for a massless real vector field,
    and at the same time, a system of Klein-Gordon equations.
    Relativistic wave equations for spinor fields satisfy certain other constraints, like the Dirac equation
    in case of massive spin $1/2$ fields, but still, every component of a spinor field have to satisfy the Klein-Gordon equation \cite{sakurai}.
          % MAYBE NOTE ON THE QUANTUM FIELDS AND HOW KG IS FUNDAMENTAL
      % Furthermore, in the context of quantum field theory it can be shown that every component of a noninteracting quantum field satisfies a Klein-Gordon equation [weinberg].
      % % Some of them, like spinor fields, satisfy additional equations as well, depending on the number of their components and
      % % independent particle states, but Klein-Gordon equation serves as a starting point of any perturbative quantum field theory [weinberg].
      % In this sense, relativistic quantum mechanics is essentially built on top of the Klein-Gordon equation.

      % The space of solutions of the Klein-Gordon equation
      % is a linear space of time-dependent square-integrable fields.
      % \begin{equation}
      %   \mathcal{V} = \left\{ \psi: \mathbb{R} \rightarrow L^2(\mathbb{R}^3) \quad | \quad \partial_t^2 \psi(x^0) -(\bigtriangleup + m^2)\psi(x^0) = 0 \right\}
      %   % + basis of plane waves
      % \end{equation}
      % % ŻE WE WILL ONLY CONCIDER COMPLEX FIELDS

     The general solution of a Klein-Gordon equation can be obtained using a Fourier transform:
     \begin{gather}
       \phi(x) = \int\frac{d^4 p}{(2\pi)^4} e^{-ipx} \tilde \phi(p)\\
       \left( \partial_\mu \partial^\mu -  m^2 \right) \phi(x) = \int\frac{d^4 p}{(2\pi)^4} e^{-ipx} \left( p_\mu p^\mu -  m^2\right) \tilde \phi(p) = 0.
     \end{gather}
     The above equation have to be satisfied for an arbitrary $\tilde \phi(p)$, which is true if and only if the $\tilde \phi(p)$
     is zero for all $p$ which do not satisfy $\left(p^0\right)^2 = E(\vec p)^2 = \vec p ^2 + m^2$. Specifically,
     $\tilde \phi(p)$ have to take the form
     \begin{equation}
       \tilde \phi(p) = A(\vec p) \delta\left((p^0)^2 - E(\vec p)^2\right),
     \end{equation}
     with an arbitrary $A(\vec p)$.
     % This means, that the basis of a space of solutions can be constructed from plane waves satisfying the relativistic dispersion relation:
     % \begin{gather}
     %   f_{\vec p} = e^{}
     % \end{gather}
     % We obtain a constraint, which has to be satisfied by the four-dimensional plane waves of all the momenta in order to solve Klein - Gordon equation:
     % \begin{equation}
     %   p_\mu p^\mu + m^2 = 0 \implies \left(p^0\right)^2 = E(\vec p)^2 = \vec p ^2 + m^2 \label{constraint}
     % \end{equation}
     The general solution for a field $\phi (x)$ will therefore be: %a combination of plane waves satisying (\ref{constraint}) with appriopriate coefficients $C(p)$:
     \begin{equation}
       \phi(x) = \int \frac{d^4p}{(2 \pi)^4} e^{-ipx} A(\vec p) \delta\left((p^0)^2 - E(\vec p)^2\right). \label{general}
     \end{equation}
     The set of four-momenta satisfying $\left(p^0\right)^2 = E(\vec p)^2$ consists of solutions with both positive and negative energy $p^0$.
     This causes problems if the plane waves are to be interpreted as eigenstates of a physical particle with four-momentum $p$.
     However, expression (\ref{general}) can equally well be rewritten with positive and negative frequency waves separately:
     \begin{equation}
       \phi(x) = \int \frac{d^4p}{(2 \pi)^4} \delta\left((p^0)^2 - E(\vec p)^2\right) \theta(p^0) \left( A(\vec p) e^{ipx} + B(\vec p) e^{-ipx} \right)
     \end{equation}
     This expression, although entirely equivalent to the previous one, suggests a different interpretation: there exist two types
     of definite-momentum states, both with positive energy, but with different forms of time evolution.
     An integral over the $p^0$ component can be performed to obtain a well-known result \cite{peskin}:%To integrate over the zeroth component of four-momentum (...)
     \begin{equation}
       \phi(x) = \int \frac{d^3p}{(2\pi)^3} \frac{1}{2 E(\vec p)} \left( A(\vec p) e^{ipx} + B(\vec p) e^{-ipx} \right) \Big|_{p^0 = E(\vec p)}.
     \end{equation}
     There, we can already see the fundamental feature of the Klein-Gordon equation and the 
     relativistic quantum mechanics as a whole - the existence of a charge degree of freedom, in addition to the usual momentum degree of freedom.
            % Klein Gordon field at an instance $x^0$ features two distinct states corresponding to any given 
      % momentum $\vec p$ - one with positive and one with negative frequency:
      % \begin{equation}
      %   \psi_{\vec p}^{\lambda} = \mathcal{N} e^{-\lambda i p x}
      % \end{equation}
      % Here $\mathcal{N}$ stands for a normalization constant. For the fields with an infinite domain, these states
      % cannot be normalized and do not belong to the space of solutions $\mathcal{V}$, but instead it should be understood
      % as a limit $L\rightarrow \infty$ of a theory defined in a finite quantization box of length $L$, where momentum labels are discrete
      % and the normalization factor is $\mathcal{N} = \frac{..}{L^3}$. Now, the general solution (\ref{general}) can be understood
      % as a decomposition of a generic Klein-Gordon field $\psi \in \mathcal{V}$ in the basis formed by the solutions with well defined momentum and charge $\{ \psi_{\vec p}^{\lambda} \}$.
      % % To speak of a proper quantum mechanical interpretation of the Klein-Gordon field, however, we need to endow the space of solutions $\mathcal{V}$ with
      % % a positive-definite, relativistically invariant inner product and an evolution operator, unitary with respect to this inner product - an issue with we have already stressed.
      % % Much of the theoretical significance of the one-particle Klein-Gordon equation can be uncovered when one understand how and under what circumstances can it be done.
      \subsection{Charge current density}
      If the field $\phi(x)$ is to be interpreted as a probability amplitude for localizing a particle at point $\vec x$ and at time $x^0$,
      there would need to exist a probability density $\rho$ and a probability current $\vec j$ satisfying a continuity equation, which ensures a conservation of total probability:
      \begin{equation}
        \partial_t \rho + \vec\nabla \cdot \vec j = 0.
      \end{equation}
      Since we discuss a relativistic theory, we have to require relativistic invariance of this equation.
      The continuity equation can be rewritten in the following way:
      \begin{gather}
        \begin{gathered}
        \partial_\mu j^\mu = 0 \label{cont-rel} \\
        j^\mu := \begin{pmatrix}
          c \rho\\
          \vec j
        \end{pmatrix}.
    \end{gathered}
      \end{gather}
      Provided that $j^\mu$ is a four-vector, (\ref{cont-rel}) will indeed be valid independent of a reference frame.
      Maintaining the simplest interpretation, that a value of the Klein-Gordon field at some point is a probability amplitude for localizing a particle at that point,
      suggests the following naive definition of $j^\mu$:
      \begin{equation}
        j^\mu := \begin{pmatrix}
          c \cdot \phi^* \phi \\
          \frac{\hbar}{2 m i} \cdot \phi^* \dvec \nabla \phi \label{naive}
        \end{pmatrix},
      \end{equation}
      where for a differential operator $D$, we define $f \dvec D g := f D g - g D f$.
      Implicitly, this interpretation also assumes that there is a set of states with a well-defined position which forms a basis of the 
      Hilbert space of the system and that the inner product of the Hilbert space is the standard inner product of non-relativistic quantum mechanics. 
      % This does not raise any issues in the ordinary quantum mechanics, but (turns out to be problematic in RQM).
      If one defines the density according to (\ref{naive}), using the plane wave ansatz it is easy to show that the continuity equation is not satisfied [ref].
      The current $j^\mu$ defined according to (\ref{naive}) is not a four-vector. 
      The definition has to be modified to ensure the Lorentz covariance of the current:
      \begin{equation}
        j^\mu := \frac{i \hbar}{2m} \ \phi^* \dvec{\partial^\mu} \phi.
      \end{equation}
      Thusly defined four-current satisfies the continuity equation for any solution of the Klein-Gordon equation:
      \begin{gather}
        \begin{gathered}
        \partial_\mu j^\mu = \frac{i \hbar}{2m} \partial_\mu \left[\phi^* \dvec{\partial^\mu} \phi \right]
        = \frac{i \hbar}{2m} \left[ \partial_\mu \phi^* \partial^\mu \phi + \phi^* ( \Box \phi ) - ( \Box \phi^* ) \phi - \partial^\mu \phi^* \partial_\mu \phi \right]\\
        = \frac{i \hbar}{2m} \left[ \mu \phi^* \phi - \mu \phi^* \phi \right] = 0.
        \end{gathered}
      \end{gather}
      The consequence of this generalized definition is that now the density $\rho(x)$ is allowed to have both positive and
      negative values, depending on the initial conditions of the Klein-Gordon field. The probability density interpretation
      cannot thus be applicable. Instead, when describing a charged particle, the current $j^\mu$ multiplied by the particle's charge
      can be interpreted as an electric charge four-current:
      %  the four vector $j^\mu$ is interpreted as the electric charge density and current.
      % The field $\phi$ is understood to have a dimension of charge and () can be rewritten using a dimensionless field $\phi = e \psi$:
      \begin{equation}
        J^\mu = e j^\mu = \frac{i \hbar e}{2m} \ \phi^* \dvec{\partial^\mu} \phi.
      \end{equation} 
      This relates the Klein - Gordon field to an observable quantity, but it is still insufficient for a quantum mechanical understanding of the Klein-Gordon equation.
      To speak of a proper quantum mechanical interpretation of the Klein-Gordon field, we need to endow the space of solutions of the Klein-Gordon equation $\mathcal{V}$ with
      a positive-definite, relativistically invariant inner product and an evolution operator unitary with respect to this inner product - an issue which we have already stressed.
      Much of the theoretical significance of the one-particle Klein-Gordon equation can be uncovered when one understands how and under what conditions it can be done.
      
      \subsection{Klein - Gordon equation in Schrodinger form}
      % In an attempt to identify the 
      The chief difference between the Schrodinger equation (\ref{sch2}) and the Klein - Gordon equation (\ref{kg}), as it stands, is the order of the differential equation.
      The existence of the additional charge degree of freedom in the latter can be attributed to this difference.
      The second order of the Klein - Gordon equation indicates that its solutions have two degrees of freedom - the initial field configuration and the initial time derivative or the field momentum.
      % This is indicateas welld  by the sign degree of freedom of the eigenfunctions $\Phi^{(\pm)}_{\vec p}$. 
      To stress this, one can express the Klein - Gordon field in terms of two components, $\eta$ and $\chi$, incorporating
      both the field configuration and field momentum. 
      The most natural definition, $\eta = \phi$ and $\chi = i \partial_t \phi$, yields equations which are
      not symmetrical and are quite difficult to work with. Instead, the more convenient symmetrical form of the equation can be obtained
      with the following definition:
      \begin{gather}
        \begin{gathered}
        \phi = \frac{ 1}{\sqrt{2}} (\eta + \chi)\\
        i \frac{\partial \phi}{\partial t} = \frac{ 1}{\sqrt{2}} (\eta - \chi). \label{component1}
        \end{gathered}
      \end{gather}
      This produces two coupled differential equations for $\eta$ and $\chi$ components:
      \begin{gather}
        \begin{gathered}
        i \frac{\partial \eta}{\partial t} = - \frac{1}{2m} \nabla^2 (\eta + \chi) + m \eta \\
        i \frac{\partial \chi}{\partial t} = \frac{1}{2m} \nabla^2 (\eta + \chi) - m \chi. \label{two_kg}
        \end{gathered}
      \end{gather}
      % That this system of equations is equivalent to the Klein - Gordon equation can be seen \dots\\
      The equations can be put in a more compact form, by defining
      \begin{equation}
        \Psi(\vec x,t) = \begin{pmatrix}
          \eta(\vec x,t) \\
          \chi(\vec x,t)
        \end{pmatrix}. \label{component2}
      \end{equation}
      % Then, (\ref{two_kg}) can be expressed with the help of the Pauli matrices $\hat\tau_i$ and the momentum operator $\hat p = -i \nabla$ as
      % \begin{gather}
      %   i \frac{\partial \Psi}{\partial t} = \hat H_0 \Psi\\ \label{kg_sch}
      %   \hat H_0 = (\hat\tau_3 + i\hat\tau_2) \frac{\hat p^2}{2m} + \hat\tau_3 m
      % \end{gather}
     Then, (\ref{two_kg}) can be expressed with the help of the Pauli matrices $\hat\tau_i$ as
     \begin{gather}
       i \partial_t \Psi(\vec x,t) = \hat H_0 \Psi(\vec x,t)\\ \label{kg_sch}
       \hat H_0 = (\hat\tau_3 + i\hat\tau_2) \frac{- \nabla^2}{2m} + \hat\tau_3 m.
     \end{gather}
     This first-order differential equation looks exceptionally similar to the Schrodinger wave equation (\ref{sch2}), with the addition of the rest mass term and the Pauli operators.
     The charge density expressed in terms of the two-component representation takes the form
     \begin{equation}
       J^0 = e \Psi^\dagger \hat \tau_3 \Psi.
     \end{equation}
     The space integral of the charge density should be equal to the total charge of a field.
     If $\Psi$ is used to describe a single particle or an antiparticle, the normalization
     condition can be formulated:
     \begin{gather}
       Q = e \int d^3x \Psi^\dagger \hat \tau_3 \Psi = \pm e \implies \int d^3x \Psi^\dagger \hat \tau_3 \Psi = \pm 1.
     \end{gather}
     This suggests that the correct form of the inner product for Klein-Gordon fields differs from the ordinary inner product
     of non-relativistic quantum mechanics. The Klein - Gordon product is defined as:
     \begin{equation}
       (\Psi | \Psi')_{\text{KG}} := \int d^3 x \Psi^\dagger \hat \tau_3 \Psi' \label{inner}.
     \end{equation}
     For the expression on the right-hand side to be convergent, it is sufficient that both the argument
     components are square integrable $\mathbb{R}^3 \rightarrow \mathbb{C}$ functions. The domain
     of the product (\ref{inner}) can therefore be defined as
     \begin{equation}
       \mathcal{V} = \left\{ \Psi \in L^2(\mathbb{R}^3) \oplus L^2(\mathbb{R}^3) \ | \ i \partial_t \Psi = \hat H_0 \Psi \right\}.
     \end{equation}
     The key remaining problem is the lack of positive-definiteness of the Klein-Gordon product (\ref{inner}).
     Due to its indefiniteness, the Klein-Gordon product does not form a proper Hilbert space over the vector space $\mathcal{V}$ and there are
     unavoidable problems with introducing a probabilistic interpretation of such a theory.
     One can proceed with describing an indefinite-metric theory and accept the Klein-Gordon equation as only
     an intermediate step to a proper framework of Quantum Field Theory.
     In fact, for a theory containing any kind of interaction, a single-particle sector of QFT cannot be separated
     and, as we shall see, in these circumstances there is no simple single-particle quantum theory\footnote{Hilbert space of an interacting Quantum Field Theory is not a free particles' Fock space \cite{weinberg}}.
     For a free Quantum Field Theory, however, single-particle description is a fundamental building block
     %  constructed from a single-particle description [ref]
     and it is crucial to understand it properly.
     % One option is to proceed with an indefinite-metric theory, abandoning probabilistic interpretation for the charge density (...).
     % Another path, studied, among others, by Mostafazadeh and Zamani is to redefine the product (\ref{inner}) in order to make it 
     % both relativistically invariant and positive-definite. 
     % In [ref] (eq. ..), they provide the expression for the most general positive definite, relativistically invariant and conserved inner product on the space
     % of solutions of the Klein - Gordon equation and construct a Hilbert space of the Klein-Gordon theory.
     We shall continue a systematic introduction of a theory based on indefinite inner product (\ref{inner}), treating
     $\Psi$ as a classical field belonging to $\mathcal{V}$, rather than a quantum mechanical wavefunction,
     and then discuss the conditions that allow us to formulate a proper probabilistic theory.      
      % The way to circumvent these problems and arrive at a proper quantum mechanical theory is to restrict the space
      % $\mathcal{V}$ to purely positive-frequency or purely negative-frequency solutions, i.e. to consider purely one-particle or one-antiparticle states.

      % Another path, studied, among others, by Mostafazadeh and Zamani is to further modify
      % the product (\ref{inner}). 
      % In [ref] (eq. ..) they provide the
      % expression for the most general positive definite, relativistically invariant and conserved inner product on the space
      % of solutions of the Klein - Gordon equation and the construction of Hilbert space over space $\mathcal{V}$.
      % The Klein-Gordon product (\ref{inner}) can be recovered as a special case
      % of the general form of inner product, when the space of states is restricted to 
      % purely positive- or purely negative-frequency solutions.
      % Here, we will further investigate the theory obtained with an unmodified product (\ref{inner}) by restricting the space of states.

      \subsection{Solutions of the Klein -Gordon equation and the Feshbach-Villard representation}
      % (ze 2 zdania jeszcze tutaj)
      Adopting the Klein-Gordon inner product implies a modified definition of the operator hermitian conjugation, denoted by $\ddagger$:
      \begin{equation}
        \Omega^\ddagger = \hat \tau_3 \Omega^\dagger \hat \tau_3
      \end{equation}
      Under this definition, operator $\hat H_0$ introduced in (\ref{kg_sch}) is a self-adjoint operator and
      it generates an evolution operator, which conserves the inner product: % Can we just say it is unitary?
      \begin{equation}
        (\hat S(t) \Psi | \hat S(t) \Psi')_{\text{KG}} = (e^{-i \hat H_0 t} \Psi | e^{-i \hat H_0 t} \Psi')_{\text{KG}} =  (\Psi | \Psi')_{\text{KG}}
      \end{equation}
      To find the eigenvectors of Hamiltonian operator $\hat H_0$, we use the following ansatz:
      \begin{equation}
        \Psi(\vec x, t) = \Psi_0(\vec p) e^{-i \lambda p x} = \begin{pmatrix}
          \eta_0(\vec p)\\
          \chi_0(\vec p)
        \end{pmatrix} e^{-i \lambda p x} 
      \end{equation}
      and insert it into the equation (\ref{kg_sch}).
      This yields the eigenvalue equation $\hat H_0 \Psi_{\vec p}^\lambda = h \Psi_{\vec p}^\lambda$ with a distinct eigenvalue for each charge and momentum $h_{\vec p}^\lambda = \lambda \sqrt{m^2+\vec p^2} \equiv \lambda E_{\vec p}$.
      With an additional condition required for the normalization of basis states to the Dirac delta, $\eta_0^2 - \chi_0^2 = 1$, corresponding eigenvectors of the Hamiltonian are:
      \begin{equation}
        \Psi^\lambda_{\vec p}(\vec x, t) = 
        \frac{1}{2 \sqrt{m E_{\vec p}}}
        \begin{pmatrix}
          m + \lambda E_{\vec p} \\
          m - \lambda E_{\vec p}
        \end{pmatrix}
        e^{-i \lambda p x}.
      \end{equation}
      They satisfy the orthogonality condition:
      \begin{equation}
        (\Psi^\lambda_{\vec p}|\Psi^{\lambda'}_{\vec q}) = \lambda \delta_{\lambda\lambda'} \delta(\vec p - \vec q)
      \end{equation}
      and as a result, a generic state $\Psi$ can be decomposed according to
      \begin{equation}
        \Psi(\vec x, t) = \sum_\lambda \int \frac{d^3p}{(2\pi)^3} \lambda (\Psi^\lambda_{\vec p}|\Psi) \cdot \Psi^\lambda_{\vec p}(\vec x, t).
      \end{equation}
      % The energy of these states is always positive and independent of $\lambda$. We can see this
      % by looking at the expression for the mean energy of a system in an eigenstate $\Psi^\lambda_{\vec p}$:
      % \begin{equation}
      %   \langle E \rangle = ( \Psi^\lambda_{\vec p}| \hat H_0 | \Psi^\lambda_{\vec p}) = \lambda E_{\vec{p}} ( \Psi^\lambda_{\vec p} | \Psi^\lambda_{\vec p}) = \lambda^2 E_{\vec{p}} \cdot \delta(0)
      % \end{equation}

      The structure of the Klein-Gordon equation is fundamentally based on the momentum and charge, rather than the position and charge.
      The difficulty of constructing states with a well-defined position and charge at the same time is a characteristic feature
      of relativistic field equations\cite{localization}. Therefore, rather than $\Psi(\vec x,t)$, it is the momentum-space field,
      \begin{equation}
      \Phi(\vec p, t) = \int d^3 x \Psi(\vec x, t) e^{-i \vec p \cdot \vec x},
      \end{equation}
      which is more natural to work with.
      Later on, we shall see how the field $\Phi(\vec p, t)$ relates to a physical wave function.
      % Rather than with a field $\Psi(\vec x,t)$, it is more natural then to work with a fourier transformed field $\Phi = \int \Psi() $.
      The Schrodinger equation for field $\Phi(\vec p,t)$ can be derived by inserting fourier decomposition of $\Psi(\vec x, t)$ in (\ref{kg_sch}) \cite{feshbach}:
      \begin{gather}
       i \partial_t \Phi(\vec p, t) = \left[ \frac{\vec{p}^2}{2m} (\hat\tau_3 + i\hat\tau_2)  + \hat\tau_3 m\right] \Phi(\vec p, t).
      \end{gather}
      The immediate result is that the Hamiltonian operator $\hat H_0$ is now diagonal in the momentum operator $\nabla_x$.
      % OGARNĄĆ JAK JEST Z WEKTORAMI BAZOWYMI TAU-SPACE PRZED ROTACJĄ I CO ONE Z POCZĄTKU ZNACZĄ
      % TAK TO JESTEŚMY JUŻ W DOMU I TYLKO ROTUJEMY W TAU, NIE W PRZESTRZENI FUNKCYJNEJ
      Eigenvectors of the Hamiltonian in momentum space are:
        \begin{equation}
        \Phi^{\lambda}_{\vec q }(\vec p, t) = 
        % \begin{pmatrix}
        %   \eta_{\vec q}^\lambda(\vec p, t) \\
        %   \chi_{\vec q}^\lambda(\vec p, t)
        % \end{pmatrix}
        % \delta(\vec p - \vec q)
        % e^{-\lambda E_{\vec{q}} \ t}
        \Phi_0(\vec q,\lambda)
        \delta(\vec p - \vec q)
        e^{-\lambda E_{\vec{q}} \ t}
        =
          \frac{1}{2 \sqrt{m E_{\vec q}}}
          \begin{pmatrix}
            m + \lambda E_{\vec q} \\
            m - \lambda E_{\vec q}
          \end{pmatrix}
          \delta(\vec p - \vec q)
          e^{-\lambda E_{\vec{q}} \ t}.
      \end{equation}
      Let us denote the standard basis vectors of the charge space as
      \begin{equation}
        e_{(+)} = 
        \begin{pmatrix}
          1 \\
          0
        \end{pmatrix} \quad
        e_{(-)} = 
        \begin{pmatrix}
          0 \\
          1
        \end{pmatrix}.
      \end{equation}
      So far, the field $\Phi$ has been expressed in terms of these vectors as
      \begin{equation}
        \Phi(\vec p, t) = \eta(\vec p,t) \ e_{(+)} + \chi(\vec p,t) \ e_{(-)}.
      \end{equation}
      In much the same way as the Fourier transform rotated the infinite-dimensional functional space
      from the position basis to the preferred momentum basis, a charge space can be rotated 
      from the basis $\{ e_{(+)},e_{(-)} \}$ to $\{\Phi_0(\vec q,+),\Phi_0(\vec q,-)\}$:
      \begin{equation}
       \bar \Phi(\vec p,t) = u(\vec p,t) \ \Phi_0(\vec p,+) + v(\vec p,t) \ \Phi_0(\vec p,-).
      \end{equation}
      This allows the Hamiltonian eigenvectors to be eigenvectors of both the momentum operator $\nabla_x$ and
      the charge sign operator $\hat \tau_3$. We shall call $\bar \Phi$ a field in the Feshbach-Villard representation.
      The momentum space field $\Phi$ in the standard representation and field $\bar \Phi$ in the Feshbach-Villard representation
      are related to each other by a linear transformation $U$ \cite{feshbach}:
            \begin{equation}
        \Phi = 
        \begin{pmatrix}
          \eta(\vec p,t) \\
          \chi(\vec p,t)
        \end{pmatrix}
        = \frac{1}{2 \sqrt{m E_{\vec q}}}
        \begin{pmatrix}
          m + \lambda E_{\vec q} & m - \lambda E_{\vec q} \\
          m - \lambda E_{\vec q} & m + \lambda E_{\vec q} \\
        \end{pmatrix}
        \begin{pmatrix}
          u(\vec p,t) \\
          v(\vec p,t)
        \end{pmatrix}
        = U(\vec p) \bar \Phi(\vec p,t).
      \end{equation}
      Thus, the Hamiltonian eigenvectors in the Feshbach-Villard representation are:
      \begin{equation}
        \Phi^{(+)}_{\vec q }(\vec p, t) =  
        \begin{pmatrix}
          1 \\
          0
        \end{pmatrix}
        \delta(\vec p - \vec q)
        e^{- E_{\vec{q}} \ t}
        \qquad
        \Phi^{(-)}_{\vec q }(\vec p, t) =  
        \begin{pmatrix}
          0 \\
          1
        \end{pmatrix}
        \delta(\vec p - \vec q)
        e^{+ E_{\vec{q}} \ t}.
      \end{equation}
      Because both the functional space basis and the charge space basis now coincide with the Hamiltonian eigenvectors,
      is is easy to confirm that the Schrodinger equation will now take on a particularly simple form:
      \begin{gather}
        i \partial_t \bar \Phi(\vec p, t) = \hat H_0 \bar \Phi(\vec p, t) = \hat \tau_3 E_{\vec p} \bar \Phi(\vec p, t).
      \end{gather}



      % Notice, that in this representation vectors with well-defined charge still are not eigenvectors of the operator $\hat \tau_3$.
      % Rather than handling the factors of $E_{\vec q}$ and $m$, it is helpful to perform a rotation in the two dimensional charge space.

      \subsection{Quantum States of the Klein-Gordon Theory}
      \begin{itemize}
        \item Comparison of one particle free QFT sector and the klein gordon equation
        \item Mostafazadeh and Zamani \cite{kghilbert} approach
      \end{itemize}

      %%%%

      % \begin{gather}
      %   (\lambda E_{\vec p} - m) \eta_0 = (p^2/2m)(\eta_0 + \chi_0)\\
      %   (\lambda E_{\vec p} + m) \chi_0 = - (p^2/2m)(\eta_0 + \chi_0)
      % \end{gather}
      % where $E_{\vec p} = \sqrt{m^2 + p^2}$. The basis of solutions consists therefore of the states

      % Comparing the form of a $\lambda = +1$ and $\lambda = -1$ solutions in the two-component representation reveals that the two
      % are related by a transformation which we will call a charge conjugation
      % \begin{equation}
      %   \Psi_C = \hat\tau_3 \Psi^*
      % \end{equation}
      % % Charge conjugation follows from comparing a (wavefunction?) with well defined momentum and positive or negative charge.
      % In the two-component representation, the functions defined in (\ref{eigenfun}) will take form
      % \begin{equation}
      %   \Psi_{\vec p}^\lambda = \begin{pmatrix}
      %     \eta^\lambda_{\vec p} \\
      %     \chi^\lambda_{\vec p}
      %   \end{pmatrix} = 
      %   \cdots = \begin{pmatrix}
      %     mc^2 + \lambda E_{\vec p} \\
      %     mc^2 - \lambda E_{\vec p}
      %   \end{pmatrix} e^{-i \lambda p x}
      % \end{equation}
      % Applying charge conjugation then results in a followig transformation:
      % \begin{gather}
      %   (\Psi_{\vec p}^\lambda)_C = \cdots = \Psi_{-\vec p}^{-\lambda}\\
      %   \rho(x) \rightarrow -\rho(x)
      % \end{gather}
      % Additionally, charge conjugation is involutive:
      % \begin{equation}
      %   (\Psi_C)_C = \Psi
      % \end{equation}
      % Thus, if the state $\Psi$ describes a particle, then $\Psi_C$ describes the corresponding antiparticle.
      % The charge conjugation can be used to split space $\mathcal{V}$ into two subspaces $\mathcal{V}_\pm$ according to
      % \begin{equation}
      %   \mathcal{V}_\pm := \{  \Psi_\pm \in \mathcal{V} \ | \ (\Psi_\pm)_C = \pm \Psi_\pm \}
      % \end{equation}
      % By restricting the space to the positive frequency states $\mathcal{V}_+$, the Klein-Gordon product becomes positive-definite.
      % A generic field in $\mathcal{V}_+$ can be decomposed in the basis of plane waves % SLOWO O TYM ELEMENCIE PSI PM VEC P
      % \begin{equation}
      %   \Psi_+ (x) = \int \frac{d^3p}{(2\pi)^3} \frac{1}{2 E(\vec p)} A(\vec p) e^{- ipx} \equiv \int d^3p \ A(\vec p) \Psi^+_{\vec p}
      % \end{equation}
      % %= \int d^3p \ A(\vec p) |\Psi^\pm_{\vec p} \rangle
      % Then, the vectors $|\Psi^+_{\vec p}\rangle$ can be interpreted as a states of a particle with well defined charge $\lambda = + 1$
      % and momentum $\vec p$. The Klein-Gordon product of these states yields
      % \begin{equation}
      %   (\Psi^+_{\vec p} | \Psi^+_{\vec q})_{\text{KG}} = \cdots = \delta(\vec p - \vec q)%\int d^3 x \Psi^\dagger \hat \tau_3 \Psi'
      % \end{equation}
      % Adopting the Klein-Gordon inner product also implies a modified definition of the hermitian conjugation of an operator, from now on denoted by $\ddagger$:
      % \begin{equation}
      %   \Omega^\ddagger = \hat \tau_3 \Omega^\dagger \hat \tau_3
      % \end{equation}
      % Under this definition, operator $\hat H_0$ introduced in (\ref{kg_sch}) is a hermitian operator and
      % the corresponding evolution operator of the differential equation (\ref{kg_sch}), $\hat S(t)$ is unitary:
      % \begin{equation}
      %   (\hat S(t) \Psi | \hat S(t) \Psi')_{\text{KG}} = (e^{-i \hat H_0 t} \Psi | e^{-i \hat H_0 t} \Psi')_{\text{KG}} = ... =  (\Psi | \Psi')_{\text{KG}}
      % \end{equation}

      % The key requirement for the consistency of this approach is the lack of mixing between the positive- and negative-frequency
      % states. If this requirement is satisfied, then an initial state $\Psi_0 \in \mathcal{V}_\pm$ remain in the space $\mathcal{V}_\pm$
      % at all times and its evolution can be consistently described by a unitary operator. As we shall see, this requirement
      % is satisfied in a noninteracting theory, but failes when a background electromagnetic potential is introduced.


      % I TUTAJ JAZDA O PRZESTRZENI STANÓW, KONIECZNOŚCI OBCIĘCIA ALBO Mostafazadeh and Zamani I DOPIERO POTEM
      % ORTONORM UNITARITY ITD JUŻ JAKO POKAZANIE ŻE W TEJ PRZESTRZENI HILBERTA WSZYSTKO GRA DOKŁADNIE JAK TRZEBA


      % Comparing the form of a $\lambda = +1$ and $\lambda = -1$ solutions in the two-component representation reveals that the two
      % are related by a transformation which we will call a charge conjugation
      % \begin{equation}
      %   \Psi_C = \hat\tau_3 \Psi^*
      % \end{equation}
      % % Charge conjugation follows from comparing a (wavefunction?) with well defined momentum and positive or negative charge.
      % In the two-component representation, the functions defined in (\ref{eigenfun}) will take form
      % \begin{equation}
      %   \Psi_{\vec p}^\lambda = \begin{pmatrix}
      %     \eta^\lambda_{\vec p} \\
      %     \chi^\lambda_{\vec p}
      %   \end{pmatrix} = 
      %   \cdots = \begin{pmatrix}
      %     mc^2 + \lambda E_{\vec p} \\
      %     mc^2 - \lambda E_{\vec p}
      %   \end{pmatrix} e^{-i \lambda p x}
      % \end{equation}
      % Applying charge conjugation then results in a followig transformation:
      % \begin{gather}
      %   (\Psi_{\vec p}^\lambda)_C = \cdots = \Psi_{-\vec p}^{-\lambda}\\
      %   \rho(x) \rightarrow -\rho(x)
      % \end{gather}
      % Additionally, charge conjugation is (zwrotna):
      % \begin{equation}
      %   (\Psi_C)_C = \Psi
      % \end{equation}
      % Thus, if the state $\Psi$ describes a particle, then $\Psi_C$ describes the corresponding antiparticle.
      % Existence of antiparticle, or of an additional charge degree of freedom in wavefunction is crucial for a proper
      % formulation of relativistic quantum mechanics.

      % Only structure, THEN observables, energy vs hamiltonian and also charge observable - charge current density
      % and problems with localization      

      %% >>> Inner product
      %% >>> Unitarity of evolution + hamiltonian is not an energy operator

      % It is now tempting to interpret the field $\Psi$ and operator $\hat H_0$
      % as the wavefunction and the Hamiltonian of the quantum mechanical system generated by the KG eq.
      % We expect $\hat H_0$ to be a hermitian operator, however it turns out not to be the case
      % \begin{equation}
      %   \hat H_0^\dagger = (\hat\tau_3 - i\hat\tau_2) \frac{\hat p^2}{2m} + \hat\tau_3 m \neq \hat H_0
      % \end{equation}
      % Instead, the following identity holds:
      % \begin{equation}
      %   \hat\tau_3 \hat H_0^\dagger \hat\tau_3 =  \hat H_0
      % \end{equation}
      % Moreover, a straightforward calculation shows, that charge density expressed in terms of $\Psi$ is
      % \begin{equation}
      %   J^0 = e \Psi^\dagger \tau_3 \Psi
      % \end{equation}
      % which implies the following condition for normalization of $\Psi$:
      % \begin{equation}
      %   \int d^3 x \Psi^\dagger \hat \tau_3 \Psi = \pm 1
      % \end{equation}
      % The seemingly problematic features of Klein Gordon equations become more clear, when we %are resolved by/coś innego ale nie to
      % realize that the inner product adequate for the KG eq. is not the standard inner product, but one
      % involving a charge conjugation:
      % \begin{gather}
      %   \Psi_C = \hat\tau_3 \Psi^*\\
      %   (\Psi | \Psi') := \int d^3 x \Psi_C^T \Psi' = \int d^3 x \Psi^\dagger \hat \tau_3 \Psi' \label{inner}
      % \end{gather}
      % Charge conjugation follows from comparing a (wavefunction?) with well defined momentum and positive or negative charge.
      % In the two-component representation, the functions defined in (\ref{eigenfun}) will take form
      % \begin{equation}
      %   \Psi_{\vec p}^\lambda = \begin{pmatrix}
      %     \eta^\lambda_{\vec p} \\
      %     \chi^\lambda_{\vec p}
      %   \end{pmatrix} = 
      %   \cdots = \begin{pmatrix}
      %     mc^2 + \lambda E_{\vec p} \\
      %     mc^2 - \lambda E_{\vec p}
      %   \end{pmatrix} e^{-i \lambda p x}
      % \end{equation}
      % Applying charge conjugation then results in a followig transformation:
      % \begin{gather}
      %   (\Psi_{\vec p}^\lambda)_C = \cdots = \Psi_{-\vec p}^{-\lambda}\\
      %   \rho(x) \rightarrow -\rho(x)
      % \end{gather}
      % Additionally, charge conjugation is (zwrotna):
      % \begin{equation}
      %   (\Psi_C)_C = \Psi
      % \end{equation}
      % Thus, if the state $\Psi$ describes a particle, then $\Psi_C$ describes the corresponding antiparticle.
      % As it was shown in [ref], the inner product (\ref{inner}) is the simpliest example of positive-definite,
      % relativistically invariant and conserved inner product on the space of solutions of the Klein-Gordon equation.
      % The Hilbert space constructed using this inner product is unitarily equivalent to $L^2(\mathbb{R})\oplus L^2(\mathbb{R})$.
      % The solutions physically correspond to states with well defined momentum and charge, and any state can be decomposed
      % as: %% Jakoś inaczej, tutaj dać bazę, rozkład jedności itp. wszystko rozjaśnić ostatecznie, hermiticity and evolution operator unitarity

      % The situation is not so simple for the position wavefunction. The value of Klein-Gordon field at a particular point
      % in space and time does not correspond to the probability amplitude for localizing the field at this point. Localization
      % schemes like the one of Newton and Wigner [ref] can be employed, but even then the position wavefunction (is different...),
      % specifically, the eigenstates of position operator are not dirac deltas, but packets of width approximately equal to
      % two de broglie wavelengths. Eventually, any attempts of constructing localized state in the relativistic quantum mechanics come into problems related to  (no-go theorems) [refs].
      % %% O wadze stopnia swobody z antycząstkami w RQM (???)

      % Observables, evolution generating operator is different than the energy observable\\

      % As we shall see, the limitations of one-particle relativistic quantum mechaincs
      % does not lie in the existence of the sign degree of freedom or the 'negative energy problem', which is easily
      % resolved by interpreting it as the charge of a particle, a feature which becomes fundamental in the RQM,
      % but rather in the problem of mixing positive- and negative-charge states. Only selected observables in the noninteracting
      % theory preserve the charge degree of freedom, while a superposition of positive- and negative-charge states can
      % hardly be considered a state of a single particle. This problem is resolved only when the KG equation is reinterpreted
      % as an equation governing a quantum field operator, acting on the states from the many-particle Fock space.
    \subsection{Klein - Gordon equation in a backround electromagnetic field}
    \begin{itemize}
      \item       Minimal coupling prescription
      \item       One - particle wavefunctions with homogenous background field
      \item       Non-unitarity, mixing of positive and negative states, Klein Paradox and the necessity of quantum fields
    \end{itemize}
      %Newton - Wigner localization, Zitterbewegung, 
      % Non-unitarity -> It is an open system -> Relativistic particle + EM field requires a field as a quantum system -> QFT
    \subsection{Quantum fields}
    \begin{itemize}
      \item       Second quantization
      \item       Second quantizaton with c-number background electromagnetic potential
      \item       Bogoliubov transformation
    \end{itemize}
  \section{Sauter-Schwinger effect}
  \section{Numerical results}
  \section{Conclusions}

\begin{thebibliography}{9}
    \bibitem{stone}
    Stone, M. H. (1930). Linear Transformations in Hilbert Space: III. Operational Methods and Group Theory. Proceedings of the National Academy of Sciences, 16(2), 172-175. doi:10.1073/pnas.16.2.172
    \bibitem{sakurai}
    Sakurai, J. J., Napolitano, J. (2020). Modern Quantum Mechanics. Cambridge University Press. doi:10.1017/9781108587280
    \bibitem{peskin}
    Peskin, M. E., Schroeder, D. V. (1995). An Introduction to Quantum Field Theory. Westview Press.
    \bibitem{weinberg}
    Weinberg, S. (1995). Quantum Theory of Fields. Press Syndicate of the University of Cambridge.
    \bibitem{feshbach}
    Feshbach, H., Villars, F. (1958). Elementary Relativistic Wave Mechanics of Spin 0 and Spin 1/2 Particles. Reviews of Modern Physics, 30(1), 24-45. doi:10.1103/revmodphys.30.24
    \bibitem{kghilbert}
    Mostafazadeh, A., Zamani, F. (2006). Quantum mechanics of Klein-Gordon fields I: Hilbert Space, localized states, and chiral symmetry. Annals of Physics, 321(9), 2183-2209. doi:10.1016/j.aop.2006.02.007
    \bibitem{qftlimit}
    Padmanabhan, T. (2018). Obtaining the non-relativistic quantum mechanics from quantum field theory: issues, folklores and facts. The European Physical Journal C, 78(7). doi:10.1140/epjc/s10052-018-6039-y
    \bibitem{localization}
    Falcone, R., Conti, C. (2023). Localization in Quantum Field Theory. doi:10.48550/arXiv.2312.15348
\end{thebibliography}

\end{document}